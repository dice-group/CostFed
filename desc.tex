% !TEX TS-program = pdflatex
% !TEX encoding = UTF-8 Unicode

% This is a simple template for a LaTeX document using the "article" class.
% See "book", "report", "letter" for other types of document.

\documentclass[11pt]{article} % use larger type; default would be 10pt

\usepackage[utf8]{inputenc} % set input encoding (not needed with XeLaTeX)

%%% Examples of Article customizations
% These packages are optional, depending whether you want the features they provide.
% See the LaTeX Companion or other references for full information.

%%% PAGE DIMENSIONS
\usepackage{geometry} % to change the page dimensions
\geometry{a4paper} % or letterpaper (US) or a5paper or....
% \geometry{margin=2in} % for example, change the margins to 2 inches all round
% \geometry{landscape} % set up the page for landscape
%   read geometry.pdf for detailed page layout information

\usepackage{graphicx} % support the \includegraphics command and options

% \usepackage[parfill]{parskip} % Activate to begin paragraphs with an empty line rather than an indent

%%% PACKAGES
\usepackage{booktabs} % for much better looking tables
\usepackage{array} % for better arrays (eg matrices) in maths
\usepackage{paralist} % very flexible & customisable lists (eg. enumerate/itemize, etc.)
\usepackage{verbatim} % adds environment for commenting out blocks of text & for better verbatim
\usepackage{subfig} % make it possible to include more than one captioned figure/table in a single float
\usepackage{amsmath}
% These packages are all incorporated in the memoir class to one degree or another...

%%% HEADERS & FOOTERS
\usepackage{fancyhdr} % This should be set AFTER setting up the page geometry
\pagestyle{fancy} % options: empty , plain , fancy
\renewcommand{\headrulewidth}{0pt} % customise the layout...
\lhead{}\chead{}\rhead{}
\lfoot{}\cfoot{\thepage}\rfoot{}

%%% SECTION TITLE APPEARANCE
\usepackage{sectsty}
\allsectionsfont{\sffamily\mdseries\upshape} % (See the fntguide.pdf for font help)
% (This matches ConTeXt defaults)

%%% ToC (table of contents) APPEARANCE
\usepackage[nottoc,notlof,notlot]{tocbibind} % Put the bibliography in the ToC
\usepackage[titles,subfigure]{tocloft} % Alter the style of the Table of Contents
\renewcommand{\cftsecfont}{\rmfamily\mdseries\upshape}
\renewcommand{\cftsecpagefont}{\rmfamily\mdseries\upshape} % No bold!

%%% END Article customizations

%%% The "real" document content comes below...

\title{Previous Formulas}
%%% \author{The Author}
\date{}

\begin{document}
\maketitle

\section{Previous version}

\begin{equation}
Card(E1 \bowtie E2) = MIN(Card(E1), Card(E2))
\end{equation}

\begin{equation}
Cost(E1 \bowtie_h E2) = Card(E1) * CRT + Card(E2) * CRT + 2 * CSQ
\end{equation}

\begin{equation}
Cost(E1 \bowtie_b E2) = Card(E1) * CRT + \frac{Card(E1)}{BSZ}* CSQ  + Card(E1 \bowtie E2) * CRT
\end{equation}

where the cost for sending a SPARQL query is CSQ, the cost for receiving a single result tuple is CRT and BSZ is the size of a bound join block.

In our experiments CSQ=50, CRT=0.02, BSZ=20

\section{Modified version}
\begin{equation}
Card(E1 \bowtie E2) = MIN(Card(E1), Card(E2)) * MVK(E1) *  MVK(E2)
\end{equation}
where  MVK(E1) and  MVK(E2) are multivalue multiplyers of E1 and E2 respectively.
\[
   MVK(E)= \left\{
     \begin{array}{@{\thinspace}l@{\thinspace}l}
       \frac{1}{\sqrt{2}}  &: \text{E is a triple pattern like ?s $<$p$>$  $<$o$>$ . }\\
       \frac{\text{triple count for given predicate}}{\text{distinct subject count}} &: \text{E is a triple pattern like ?s $<$p$>$  ?o. The join variable is ?s.} \\
       \frac{\text{triple count for given predicate}}{\text{distinct object count}} &: \text{E is a triple pattern like ?s $<$p$>$  ?o. The join variable is ?o.} \\
       \text{1} &: \text{other cases}
     \end{array}
   \right.
\]

\begin{equation}
Cost(E1 \bowtie_h E2) = \frac{1 + TC}{TC} * CSQ + Card(E2) * CRT + (Card(E1)  +  Card(E2)) * CHT
\end{equation}
where Card(E1) $<$ Card(E2), TC is the number of threads used to query sparql endpoints, CSQ is the cost of sending a SPARQL query, CRT is the cost of receiving a single result tuple, CHT is the cost of handling received tuple.\\\\
Description: the hash join algorithm sends 2 requests for E1 and E2 using TC threads (cost = first summand in the (5)), then recieves results for E1 and E2 in parallel, so cost = Max(Card(E1), Card(E2)) * CRT = Card(E2) * CRT, finally all the tuples received are
handled: the internal implementaion uses hashmap with synchronized access to store data, so cost can be estimated as (Card(E1)  +  Card(E2)) * CHT

\begin{equation}
Cost(E1 \bowtie_b E2) = CSQ + Card(E1) * CRT + \frac{\frac{Card(E1) + BSZ - 1}{BSZ} + CTC - 1}{CTC} * CSQ
\end{equation}

The bind join algo at first sends the request for E1 (cost = CSQ), receives results for E1 (cost = Card(E1) * CRT), then using TC threads sends resuests for E2 using bunch of BSZ size (cost = third summand in the formula (6)) \\\\
In our experiments CSQ=100, CRT=0.01, 0.0025, BSZ=20, TC=20

\end{document}



